One of the core debates of economic theory is that of the rationality of agents. Keynesian economics was the dominant school of thought until the 1970s which saw the rise of New Classical economics\autocite{Hartley2013}. Lucas and Sargent were two key spearheads of this movement with their influential paper titled "After Keynesian Macroeconomics"\autocite{Lucas1979}, key to their complaints were the lack of microeconomic foundations in Keynesian models. Real business cycle theory, a major branch of New Classical economics believed that business cycles were a rational response to exogenous shocks to the economy and were perfectly efficient. However, this also meant that business cycles would not arise endogenously as firms and consumers would learn over time the behavior of the economy as they, as a whole, acted rationally. Though Keynesianism has since gained in popularity again, these lessons of New Classical Economics continue to influence the microfoundations of macroeconomic New Keynesian models.

Westerhoff, et al. directly based their inventory cycle model on that of 