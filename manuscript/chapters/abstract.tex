\thispagestyle{plain}
\begin{center}
	\vspace*{\fill}
	\textbf{Abstract}\\
\end{center}
	Dynamical systems are present in a variety of fields including the study of economics, physics and chemistry. The study of stochastic dynamical systems was pioneered by Louis Bachelier\autocite{Bachelier1900} who attempted to model Brownian motion to study financial systems although it was later popularized by Einstein and developed into the theory of statistical mechanics. Econonomic and financial models quickly began adapting these theories into their own models with the field of econophysics being developed in order to accomodate these and related ideas. Many notable processes in chemistry and physics are represented as chaotic dynamical systems; however, chaos is still not a commonplace point of discussion in macroeconomic theory. In this paper, I present two previously created business cycle models, a multiplier-accelerator model with a Robertson lag by T\"onu Puu and an inventory cycle model with a Lundberg lag by Metzler and Wegener, before creating a new model that incorporates these elements from these two models. This model explains the inability for firms to accurately determine future outcomes on the potentially chaotic or quasi-periodic behavior of output levels. I find that even given reasonable parameter choice, it is reasonable for the model to be in a chaotic regime, thus allowing business cycles to be driven endogenously as opposed to through external, stochastic shocks.
	\vspace*{\fill}
\pagebreak