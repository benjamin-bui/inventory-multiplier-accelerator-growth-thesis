\thispagestyle{plain}
\begin{center}
	\vspace*{\fill}
	\textbf{Abstract}\\
	The field of chemistry and physics has made many contributions to economic theory in the past. The development of the theory of Brownian motion and other stochastic processes proved revolutionary in economics and finance with an entire subfield of econophysics being developed in order to accommodate these new ideas. Many notable processes in chemistry and physics are represented as chaotic dynamical systems; however, chaos is still not a commonplace point of discussion in macroeconomic theory. In this paper, I present two previously created business cycle models, a multiplier-accelerator model by T\"onu Puu and an inventory cycle model by Metzler and Wegener. I then develop a new model that incorporates the Lundberg lag and endogenous investment change of the multiplier-accelerator model and the Robertson lag driven inventory cycles of Metzler and Wegener. This model explains the inability for firms to accurately determine future outcomes on the potentially chaotic or quasi-periodic behavior of output levels. I find that even given reasonable parameter choice, it is reasonable for the model to be in a chaotic regime, thus allowing business cycles to be driven endogenously as opposed to through external, stochastic shocks.
	\vspace*{\fill}
\end{center}
\pagebreak