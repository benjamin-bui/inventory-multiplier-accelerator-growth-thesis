We can define the growth rate $\dot Y_t$ as the difference in income between contiguous time periods:
\begin{equation}
    \dot Y_t=Y_{t+1}-Y_t
\end{equation}
Let this dot notation refer to the analogous growth rate of other state variables. This equation can be expanded using Equation \ref{eq:Income}.
\begin{equation*}
    \dot Y_t=(U_{t+1}-U_t)+(I_{t+1}-I_t)+(S_{t+1}-S_t)=\dot U_t+\dot I_t+\dot S_t
\end{equation*}

The change in investment can be trivially solved for as investment is already a function of the change in income.
\begin{equation*}
    I_t=\frac{\frac{\dot Y_{t-2}}{v}}{\left(\frac{\dot Y_{t-2}}{v}\right)^4+q}
\end{equation*}
Thus the change in investment is:
\begin{equation}
    \dot I_t=\frac{\frac{\dot Y_{t-1}}{v}}{\left(\frac{\dot Y_{t-1}}{v}\right)^4+q}-\frac{\frac{\dot Y_{t-2}}{v}}{\left(\frac{\dot Y_{t-2}}{v}\right)^4+q}
\end{equation}

As predicted consumption is a function of the change in actual consumption, we must derive a function for the change in consumption in terms of the growth rate.
\begin{equation*}
    C_{t+1}-C_t=[(1-s)Y_{t}+sY_{t-1}]-[(1-s)Y_{t-1}+sY_{t-2}]
\end{equation*}
This can be simplified to:
\begin{equation}
    \dot C_t=(1-s)(Y_{t}-Y_{t-1})+s(Y_{t-1}-Y_{t-2})=(1-s)\dot Y_{t-1} + s\dot Y_{t-2}
\end{equation}

The change in expected consumption can be given by:
\begin{align*}
    U_{t+1}-U_{t}&=\frac{C_{t}+C_{t-1}+C_{t-2}}{3}-\frac{C_{t-1}+C_{t-2}+C_{t-3}}{3}\\
    \dot U_t&=\frac{\dot C_{t-1}+\dot C_{t-2}+\dot C_{t-3}}{3}
\end{align*}
This can be clearly resolved into a function of income growth rates:
\begin{equation*}
    \dot U_t=\frac{[(1-s)\dot Y_{t-2} + s\dot Y_{t-3}]+[(1-s)\dot Y_{t-3} + s\dot Y_{t-4}]+[(1-s)\dot Y_{t-4} + s\dot Y_{t-5}]}{3}
\end{equation*}
\begin{equation}
    \dot U_t=\frac{1}{3}\left((1-s)(\dot Y_{t-2}+\dot Y_{t-3}+\dot Y_{t-4})+s(\dot Y_{t-3}+\dot Y_{t-4}+\dot Y_{t-5})\right)
\end{equation}

The final step is to solve for inventory production in terms of growth rates. We can substitute our $S_t$ into our function for $Q_t$:
\begin{align*}
    Q_t&=Q_{t-1}+(kU_t-Q_{t-1}+(U_t-C_t)\\
    Q_t&=kU_t+U_t-C_t
\end{align*}
This gives an equation for the rate of change:
\begin{align*}
    Q_{t+1}-Q_{t}&=\left[kU_{t+1}+U_{t+1}-C_{t+1}\right]-\left[kU_{t}+U_{t}-C_{t}\right]\\
    \dot Q_t&= (k+1)(U_{t+1})-(k+1)(U_t)-(C_{t+1}-C_t)\\
    \dot Q_t&= (k+1)(U_{t+1}-U_t)-(C_{t+1}-C_t)\\
    \dot Q_t&=(k+1)(\dot U_t)-(\dot C_t)
\end{align*}
Thus giving a function with respect to income growth rate:
\begin{equation}
\begin{split}
    \dot Q_t& = \frac{(k+1)}{3}\left((1-s)(\dot Y_{t-2}+\dot Y_{t-3}+\dot Y_{t-4})+s(\dot Y_{t-3}+\dot Y_{t-4}+\dot Y_{t-5})\right)\\
    &-(1-s)\dot Y_{t-1}-s\dot Y_{t-2}
\end{split}
\end{equation}

The difference in production of goods for inventory is given by:
\begin{align*}
    S_{t+1}-S_{t}& =(kU_{t+1}-Q_{t})-(kU_{t}-Q_{t-1})\\
    \dot S_t &= k(U_{t+1}-U_t)+Q_{t-1}-Q_t\\
    \dot S_t& = k(\dot U_t)-(\dot Q_{t-1})
\end{align*}
We can thus substitute our functions for the change in predicted consumption and change in inventory:
\begin{equation*}
\begin{split}
    \dot S_t& =\frac{k}{3}\left((1-s)(\dot Y_{t-2}+\dot Y_{t-3}+\dot Y_{t-4})+s(\dot Y_{t-3}+\dot Y_{t-4}+\dot Y_{t-5})\right)-\\
    &\left[\frac{(k+1)}{3}\left((1-s)(\dot Y_{t-3}+\dot Y_{t-4}+\dot Y_{t-5})+s(\dot Y_{t-4}+\dot Y_{t-5}+\dot Y_{t-6})\right)\\
    &\left.-(1-s)\dot Y_{t-2}-s\dot Y_{t-3}\right]
\end{split}
\end{equation*}
\begin{equation}
\begin{split}
    \dot S_t& =\frac{k}{3}\left((1-s)(\dot Y_{t-2}+\dot Y_{t-3}+\dot Y_{t-4})+s(\dot Y_{t-3}+\dot Y_{t-4}+\dot Y_{t-5})\right)-\\
    &\frac{(k+1)}{3}\left((1-s)(\dot Y_{t-3}+\dot Y_{t-4}+\dot Y_{t-5})+s(\dot Y_{t-4}+\dot Y_{t-5}+\dot Y_{t-6})\right)\\
    &+(1-s)\dot Y_{t-2}+s\dot Y_{t-3}
\end{split}
\end{equation}

This means the function for the growth rate of the economy can be written as a 6th order, single-variable difference equation:

\begin{equation*}
\begin{split}
    \dot Y_{t}& = \frac{\frac{\dot Y_{t-1}}{v}}{\left(\frac{\dot Y_{t-1}}{v}\right)^4+q}-\frac{\frac{\dot Y_{t-2}}{v}}{\left(\frac{\dot Y_{t-2}}{v}\right)^4+q} + \\
    & \frac{(1-s)(\dot Y_{t-2}+\dot Y_{t-3}+\dot Y_{t-4})+s(\dot Y_{t-3}+\dot Y_{t-4}+\dot Y_{t-5})}{3} + \\
    &k\frac{(1-s)(\dot Y_{t-2}+\dot Y_{t-3}+\dot Y_{t-4})+s(\dot Y_{t-3}+\dot Y_{t-4}+\dot Y_{t-5})}{3}-\\
    &\left[(k+1)\frac{(1-s)(\dot Y_{t-3}+\dot Y_{t-4}+\dot Y_{t-5})+s(\dot Y_{t-4}+\dot Y_{t-5}+\dot Y_{t-6})}{3}\right.\\
    &\left.-(1-s)\dot Y_{t-2}-s\dot Y_{t-3}\right]
\end{split}
\end{equation*}
\begin{equation*}
\begin{split}
    \dot Y_{t}& = \frac{\frac{\dot Y_{t-1}}{v}}{\left(\frac{\dot Y_{t-1}}{v}\right)^4+q}-\frac{\frac{\dot Y_{t-2}}{v}}{\left(\frac{\dot Y_{t-2}}{v}\right)^4+q} + \\
    & (k+1)\left[\frac{(1-s)(\dot Y_{t-2}+\dot Y_{t-3}+\dot Y_{t-4})+s(\dot Y_{t-3}+\dot Y_{t-4}+\dot Y_{t-5})}{3} -\right.\\
    & \left.\frac{(1-s)(\dot Y_{t-3}+\dot Y_{t-4}+\dot Y_{t-5})+s(\dot Y_{t-4}+\dot Y_{t-5}+\dot Y_{t-6})}{3}\right]+(1-s)\dot Y_{t-2}+s\dot Y_{t-3}
\end{split}
\end{equation*}
\begin{equation}
\begin{split}
    \dot Y_{t}& = \frac{\frac{\dot Y_{t-1}}{v}}{\left(\frac{\dot Y_{t-1}}{v}\right)^4+q}-\frac{\frac{\dot Y_{t-2}}{v}}{\left(\frac{\dot Y_{t-2}}{v}\right)^4+q} + \\
    & \frac{k+1}{3}\left[(1-s)(\dot Y_{t-2}-Y_{t-5})+s(\dot Y_{t-3}-Y_{t-6})\right]+(1-s)\dot Y_{t-2}+s\dot Y_{t-3}
\end{split}
\end{equation}

