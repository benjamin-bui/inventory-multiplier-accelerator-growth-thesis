The study of dynamical systems is traditionally thought to have begun with the publication of "New methods of Celestial Mechanics" by Poincar\'e and expanded with the work of Lyapunov into a theory of the stability of dynamical systems. It was not until the 1960s however that the use of chaos and stability theory exploded across disciplines\autocite{Aubin2002}. 

Dynamics are typically an unnecessary tool when studying the paths of chemical reactions. Their value became apparent however, when Belousov and later Zhabotinsky released published their work on an oscillatory reaction, a reaction that would later come to be known as the BZ reaction\autocite{Winfree1984}. Cycles were long known to exist in the biochemical realm with many famous pathways in organisms such as the Krebs cycle and Calvin cycle; however the BZ reaction was developed to create an inorganic analogue to the Krebs cycle. The development of this cycle allowed for a relatively easily replicable cyclic reaction with with easily measurable indicators of the progress of the reaction. 

The study of dynamical chemical systems has since expanded to a variety of other mechanisms such as self-replicating molecules in addition to studying fractal patterns and dimensionality involved in electrochemical deposits and flame patterns. Despite the fairly wide range of background information required to set up these different models, the underlying mathematical theory used to study these models is identical which has contributed to the wide range of interdisciplinary work performed by theorists in the field.
\section{Mathematical Background}
Traditional dynamical systems are modelled in continuous time as a system of ordinary differential equations. These functions  b 
\section{Motivation}
\lipsum[1-5]
