The study of dynamical systems is traditionally thought to have begun with the publication of "New methods of Celestial Mechanics" by Poincar\'e and expanded with the work of Lyapunov into a theory of the stability of dynamical systems. It was not until the 1960s however that the use of chaos and stability theory exploded across disciplines\autocite{Aubin2002}. 

Dynamics are typically an unnecessary tool when studying the paths of chemical reactions. Their value became apparent however, when Belousov and later Zhabotinsky released published their work on an oscillatory reaction, a reaction that would later come to be known as the BZ reaction\autocite{Winfree1984}. Cycles were long known to exist in the biochemical realm with many famous pathways in organisms such as the Krebs cycle and Calvin cycle; however the BZ reaction was developed to create an inorganic analogue to the Krebs cycle. The development of this cycle allowed for a relatively easily replicable cyclic reaction with with easily measurable indicators of the progress of the reaction. 

The study of dynamical chemical systems has since expanded to a variety of other mechanisms such as self-replicating molecules in addition to studying fractal patterns and dimensionality involved in electrochemical deposits and flame patterns. Despite the fairly wide range of background information required to set up these different models, the underlying mathematical theory used to study these models is identical which has contributed to the wide range of interdisciplinary work performed by theorists in the field.
\section{Background on Dynamical Systems}
Traditional dynamical systems are modelled in continuous time as a system of ordinary differential equations. These systems typically treat time as the singular independent variable and solve for the evolution of one or more variables in terms of time. A classic continuous time dynamical system is the simple pendulum which allows one to model the movement of a pendulum in space in terms of time. This model uses a variety of simplifying assumptions in order to reduce the problem of the pendulum into a single variable function, holding the length of the pendulum and acceleration due to gravity constant.
\begin{equation}
    \frac{d^2}{dt^2}+\frac{g}{L}\sin\theta=0
\end{equation}
Most  attempts  at  modelling  real-world  systems  however  require  the  use  of  multiple dependent variables in order to effectively model.  However, as the amount of variables increases, the complexity of the model increases.  Modelling and $n$-body system acting on each other gravitationally is of obvious interest in astronomy; however, it was quickly found  that  although  a  1  and  2  body  system  were  relatively  simple  to  solve  for,  the introduction of a third body caused significant complications.  The 3-body system is in fact what Poincar\'e studied in order to develop a theory of chaotic deterministic systems\autocite{Poincare1993}.

That is not to say that these systems cannot arrive at ordered solutions.  Although continuous time systems require a minimum of 3 variables in order for chaos to arise, there are typically windows of order in chaotic regimes that allow for stable, oscillatory behavior. The BZ-reaction is still being actively studied and is known to be actually highly complex but is often reduced into 7 primary sub-reactions\autocite{Field1986}. A great deal of the work involved in studying dynamical systems is actually on finding ways to simplify models in order to arrive at more mathematically tractable systems. The BZ-reaction has been simplified to a 3-variable system that still provides the complex periodicity and chaotic behavior characteristic of the model\autocite{Gyorgyi1992}.

\section{Discrete Time Dynamical Systems}
Although many processes in the real-world are more intuitively interpreted as operating as a continuous time  function, there are many  occasions where it is possible and  in fact beneficial to think in a discrete-time sense. The prevalence of this type of system varies depending on the exact field of study; however, it is important to note that when computationally modelling continuous time systems, it is impossible for computers to truly operate with continuous variables and thus even these systems are reduced to technically discrete models.

Population dynamics are frequently analyzed as a discrete time system as opposed to continuous time. It is often of more practical use to interpret $t= 0,1,2,3,...$ as the change in population per year or per season as opposed to determining the change in population over infinitesimally small changes in time. In terms of technique, many of the mathematical principles used in analyzing dynamic systems in continuous time apply to discrete time systems; however, it would be a mistake to assume the two were identical. An important distinction between the two is the nature by which chaos can occur. As described previously, a continuous time system requires 3 or more dependent variables in order for chaos to occur. A discrete time system only requires 1 variable in order to display the same type of chaotic behavior.

The systems discussed throughout this paper will be of the discrete variety due to their nature. Laws pertaining to the physical world are scalable to the infinitesimal degree which allows for their use in continuous models. Economic models do not have a basis in physical laws. It is also important to note that, due to the complexity involved in the human behavior that economic models are trying predict, the exact numerical values of the model are typically of minor concern. The general behavior of the model is significantly more valuable in order to determine the effects of an economic assumption.

\section{The Logistic Map, A Mathematical Introduction}
The logistic map is regarded as the prototypical chaotic discrete time mapping. The logistic function, which the logistic map is based off of, was developed to study population dynamics but actually garnered widespread use in other disciplines such as the study of autocatalytic reactions, computer science, statistics, and economics\autocite{Kavanagh1934}.
\begin{equation}
    \frac{d}{dx}f(x)=f(x)(1-f(x))
\end{equation}
The logistic function has 2 equilibria or points where the derivative of the function is 0. $f(x) = 0$ is an unstable equilibrium but $f(x)=1$ is a stable equilibrium point which means that other points on the function will tend towards this equilibrium overtime.  This can be realized by solving for the derivative of the function at points when $f(x)\in(0,1)$ which is universally positive and $f(x)\in(1,\infty)$ which is negative. Integrating the differential equation gives the general form equation:
\begin{equation}
    f(x)=\frac{e^x}{e^x+C}
\end{equation}
This function gained prominence due to its rapid, exponential growth when $f(x)$ is low and its slow, linear decaying to non-existent growth as population increases.  Used by notable mathematicians operating in the field of population dynamics such as Verhulst, Pearl, and Lotka, the model continues to be widely used today and is often the basis upon which other modifications are applied\autocite{Zwanzig1973}.

\begin{equation}
    x_{t=1}=\mu x_t(1-x_t)
\end{equation}
The logistic map is a difference equation model popularized by Robert May as a discrete time analogue to the logistic function\autocite{May1976}. When interpreted in the biological context, $x_t$ refers to the ratio of the population at time $t$ compared to the maximal population, thus the mapping is bounded between 0 and 1. Here we see intuitively the major difference between difference equation and differential equation based systems. Differential equations solve for the derivative of a variable with respect to time in terms of the variable, difference equations solve for the actual state of the variable in the successive state provided we know the state of the variable in the previous time period. Much like how differential equations can be of higher order with the introduction higher order derivatives, a difference equation can also be of higher order by including more time periods in the function for the state of the variable which is valuable in a variety of the models discussed later.

Much like how the equilibrium points were solved for in the differential equation, difference equations also have equilibrium points where $x_{t+1}=x_t$. Interestingly, unlike the logistic function, there does not exist a fixed point at $x_t=1$ as this would result in $x_{t+1}=0$. Solving for the fixed points, we have:
\begin{equation}
    x_{t+1}=x_t=0,\frac{\mu-1}{\mu}
\end{equation}

The stability of a fixed point is again dependent on the derivative of the function; however, there are differences in the details of our analysis. Treating $f(x)=x_{t+1}$, we see the derivative of the logistic map is:
\begin{equation}
    f^\prime(x)=\mu(1-2x)
\end{equation}
For reasons that will become clear later, the stability of a point on the map requires that $|f^\prime(x)|<1$. Solving for when this is true for our two fixed points, we see that $\mu<1$ provides stability for the origin fixed point and $1<\mu<3$ gives stability for the non-zero fixed point. Thus, provided the parameter $\mu$ satisfies either of the conditions set previously, it will converge to one of the fixed points in a relatively small, finite amount of iterations. 

This behavior can be visualized using a cobweb diagram. This diagram consists of 3 primary elements: a plot of the mapping, a \SI{45}{\degree} line, and a plot of the variable's trajectory. An example of a cobweb diagram can be seen in Figure \ref{log_fixed_cob}. This diagram shows the trajectory of $x$ starting at a value of 0.1 when there is a stable, non-trivial fixed point.

\begin{figure}
    \centering
    \includegraphics[height=0.4\textheight]{log_fixed_cob.eps}
    \caption{Cobweb plot of the logistic map setting $\mu=1.5$ and $x_0=0.1$. The trajectory asymptotically approaches the equilibrium point of $\frac{1}{3}$.}
    \label{log_fixed_cob}
\end{figure}
The \SI{45}{\degree} line is defined as the line where $x_t=x_{t+1}$ which is useful for determining the result of successive iterations. Beginning from the point $x_0$, we can then determine what point $x_1$ wll be via the mapping. We can then look horizontally to the \SI{45}{\degree} line until we intersect with it. The $x$-coordinate of this intersection point is equivalent to the result of the mapping of the previous iteration, thus using this new point will allow us to determine the result of the next iteration of the function. This process can be repeated ad infinitum; however, the result will soon prove uninteresting for stable points and orbits as the trajectory will converge and repeat its behavior.

The reason the logistic map is so frequently studied is because of its ability to exhibit complex behavior beyond a stable equilibrium solution. Once $\mu>3$, the mapping enters a cyclic region. Much like how fixed points could be solved for by identifying where $x_{t}=x_{t+1}$, stable oscillatory points can be found by solving for the equilibrium points of higher iterations of the function. A 2-cycle will be such that $x_{t}=x_{t+2}\neq x_{t+1}$ for example and the stability of a such a cycle can be found using the same methodology as described previously. 

The logistic map also allows us to more precisely determine what it means for a mapping to be chaotic. Intuitively, a system is chaotic if the result of the system is highly dependent on its initial conditions. 

