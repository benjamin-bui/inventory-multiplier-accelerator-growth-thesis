\documentclass{beamer}
\mode<presentation>
\usepackage[utf8]{inputenc}

\usetheme{Dresden}
\usecolortheme{beaver}
\setbeamertemplate{footline}[frame number]
\setbeamertemplate{caption}[numbered]

\usepackage[english]{babel}%Set language as english%
\usepackage[babel]{csquotes}%Nicer quotations%
\usepackage{indentfirst}
\usepackage[protrusion=true,expansion=true]{microtype}
\usepackage{enumitem}%Different enumeration options%
%Math Formatting%
\usepackage{amsmath}%AMS Math%
\usepackage{amsthm}%Theorem Formatting%
\usepackage{amssymb}%AMS Symbols%
\usepackage{array}%Allows for more complex matrices
\usepackage{siunitx}%Formats SI units%
\DeclareSIUnit{\molar}{M}%Adds molarity as unit%
\usepackage{booktabs}%Typsets tables%
\newcommand{\specialcell}[2][c]{\begin{tabular}[#1]{@{}l@{}}#2\end{tabular}}
\newcommand{\specialcellbold}[2][c]{%
	\bfseries
	\begin{tabular}[#1]{@{}l@{}}#2\end{tabular}%
}
\usepackage{chemformula}%Typsets Chemical Formulas$
%Image Formatting%
\usepackage{graphicx}%Enchanced graphics support%
\usepackage{float}%Float options$


\usepackage[
backend=biber,
style=chem-acs,
articletitle
]{biblatex}
\addbibresource{references.bib}

\usetheme{default}

\title{Dynamics of Discrete Time Economic Growth Models}
\author{Benjamin Bui}
\date{October 5, 2018}
\begin{document}
\begin{frame}
	\titlepage
\end{frame}

\begin{frame}
	\frametitle{Overview}
	\tableofcontents
\end{frame}

\section{Introduction}
\begin{frame}{Motivation}
	Can we apply techniques used in studying chemical oscillations and physical dynamical systems to analyze economic models?
\end{frame}

\begin{frame}{Key Differences}
	\begin{itemize}
		\item
			Models in chemistry are based on interactions based in physical laws
		\pause
		\item
			Macroscale physical systems are in continuous time
	\end{itemize}
\end{frame}

\section{The Logistic Map}

\begin{frame}{Applications}
	\begin{itemize}
		\item
		 	Popularized by the ecologist Robert May (Cite Later)
		\item
			Used to model population dynamics
		\item
			Frequently analyzed and modified to study iterated maps
	\end{itemize}
\end{frame}

\begin{frame}{Behavior}
	\begin{columns}
		\begin{column}{0.5\textwidth}
			\begin{equation*}
				x_{t+1}=\mu(1-x_t)x_t
			\end{equation*}
		\end{column}
		\begin{column}{0.5\textwidth}
			Insert plot of logistic map
		\end{column}
	\end{columns}
\end{frame}

\begin{frame}{Stability of Equilibrium Points}
	\begin{columns}
		\begin{column}{0.5\textwidth}
			\begin{itemize}
				\item
					\textbf{Fixed Points}
				\item
					\begin{equation*}
						x=0
					\end{equation*}
				\pause
				\item
					\begin{equation*}
						\frac{\mu-1}{\mu}
					\end{equation*}
			\end{itemize}
		\end{column}
		\pause
		\begin{column}{0.5\textwidth}
			\begin{itemize}
				\item
					\textbf{Conditions for Stability}
				\item
					\begin{equation*}
						|\mu|<1
					\end{equation*}
				\pause
				\item
					\begin{equation*}
						1<\mu<3
					\end{equation*}
			\end{itemize}
		\end{column}
	\end{columns}
\end{frame}

\begin{frame}{Stability}
	\begin{columns}
		\begin{column}{0.5\textwidth}
			Stability means there is a convergence to the fixed point
		\end{column}
		\begin{column}{0.5\textwidth}
			Include figure of convergence.
		\end{column}
	\end{columns}
\end{frame}

\begin{frame}{Double Iteration}
	\begin{columns}
		\begin{column}{0.5\textwidth}
			\begin{itemize}
				\item
					\textbf{The Map}
				\item
					\begin{equation*}
						\mu^2(\mu x^2-\mu x+1)(1-x)x
					\end{equation*}
				\pause
				\item
					\textbf{Conditions of Stability}
				\item
					\begin{equation*}
						3<\mu<1+\sqrt{6}\approx3.449
					\end{equation*}
			\end{itemize}
		\end{column}
		\begin{column}{0.5\textwidth}
			Include figure of 2 cycle
		\end{column}
	\end{columns}
\end{frame}

\begin{frame}{Chaos}
	\begin{columns}
		\begin{column}{0.5\textwidth}
			\begin{itemize}
				\item
					\textbf{Lyapunov Exponent}
				\item
					\begin{equation*}
						\lambda_n(x_0)=\frac{1}{n}\sum\limits_{t=1}^{t=n}\ln\lvert f^\prime(x_{t-1})\rvert
					\end{equation*}
				\pause
				\item
					\textbf{Bifurcation Diagram}
				\item
					Visualize transition from a singular fixed point to chaos.
			\end{itemize}
		\end{column}
		\pause
		\begin{column}{0.5\textwidth}
			Show chaotic mapping
		\end{column}
	\end{columns}
\end{frame}
\end{document}
